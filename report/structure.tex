\section{Структура проекта}
Для решения проблемы был разработан проект. Он представляет собой компьютерную
программу, написанную на языке высокого уровня C. Программа последовательно
применяет различные алгоритмы сортировки к одному и тому же массиву данных
и замеряет время работы каждого в секундах.\par
Проект имеет следующую структуру:
\begin{itemize}
    \item{\textbf{report} - каталог, содержащий настоящий отчет в формате \LaTeX}
    \item{\textbf{sample} - каталог, содержащий тестовые данные}
    \item{\textbf{src} - каталог, содержащий исходный код программы}
    \begin{itemize}
      \item{\textbf{algorithms.c} - библиотека, содержащая алгоритмы сортировки. Её код приводится в Приложении}
      \item{\textbf{algorithms.h} - заголовочный файл библиотеки алгоритмов}
      \item{\textbf{main.c} - код, отвечающий за интерфейс и соновную логику работы программы}
      \item{\textbf{utils.c} - некоторые второстепенные функции}
      \item{\textbf{utils.h} - заголовочный файл}
    \end{itemize}
\end{itemize}

Для сборки проекта в системах macOS и Linux необходимо установить библиотеку ncurses и компилятор gcc
и выполнить в корне проекта следующую команду:
\begin{lstlisting}[numbers=none,language=bash]
gcc -o lab src/algirithms.c src/utils.c src/main.c -lncurses -lmenu
\end{lstlisting}
Полный код проекта доступен по адресу https://github.com/alexbat98/lab1-sort-comp