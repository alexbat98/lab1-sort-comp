\section{Введение}
Переразмещение элементов в порядке возрастания или убывания - задача, которая 
очень часто возникает в программировании. От порядка размещения данных в памяти
компьютера зависит не только удобство работы с этими данными, но и скорость выполнения
и простота алгоритмов, предназначенных для их обработки.\par
По оценкам производителей компьютеров в 60-х годах в среднем более четверти машинного времени
тратилось на сортировку. Во многих вычислительных системах на нее уходит больше половины
машинного времени~\cite{Knuth3}. Вот некоторые из наиболее распространенных областей
применения сортировки:
\begin{enumerate}
    \item Решение задачи группирования, когда нужно собрать вместе все элементы с одинаковыми значениями признака. 
    \item Поиск общих элементов в двух или более массивах. 
    \item Поиск информации по значениям ключей.
\end{enumerate}\par
При разработке программных продуктов важным этапом становится тестирование, цели которого\cite{testing:website}:
\begin{enumerate}
    \item Продемонстрировать разработчикам и заказчикам, что программа соответствует требованиям
    \item Выявить ситуации, в которых поведение программы является неправильным, нежелательным или не соответствующим спецификации
\end{enumerate}
