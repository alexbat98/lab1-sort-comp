\section{Описание эксперимента}
В программе существует 2 вида наборов данных: частично упорядоченный
и абсолютно случайный. Каждый набор содержит не менее 30 000 элементов.
Тестирование на меньшем объеме данных не имеет смысла ввиду высокой
вычислительной мощности современных компьютеров. Для чистоты эксперимента
все тесты проводились на одном и том же компьютере с процессором Intel Core i5
2,6ГГц. Все остальные приложения были закрыты, чтобы не мешать работе тестового.\par
Частично упорядоченные наборы имеют следующую структуру: первые 100 элементов содержат
случайные числа от 1 до 100, следующие 100 от 101 до 200 и так далее. Случайные
наборы просто содержат случайные числа из диапазона от 0 до N.\par
Современные компьютерные системы стараются генерировать случайные числа с нормальным
распределением. За счет этого мы сможем добиться наиболее честных условий для работы
разных алгоритмов.