\section{Обсуждение результатов}
В ходе исследования был проведен эксперимент, с помощью которого были
выявлены особенности работы алгоритмов сортировки. Результаты представлены в 
таблицах ниже.
\begin{table}[h]
\caption{Частично упорядоченные данные}
\centering
  \begin{tabular}{ | l | l | l | l | l | l | l |}
  \hline
    Алгоритм & 30k & 50k & 100k & 150k & 200k & 300k \\ \hline
    Пузырьковая сортировка & 0,017 & 0,027 & 0,054 & 0,077 & 0,099 & 0,161 \\ \hline
    Шейкерная сортировка & 0,011 & 0,018 & 0,038 & 0,055 & 0,077 & 0,122 \\ \hline
    Сортировка Шелла & 0,007 & 0,012 & 0,029 &  0,039 & 0,052 & 0,083 \\ \hline
    Сортировка вставками & 0,003 & 0,006 & 0,008 & 0,013 & 0,015 & 0,025 \\ \hline
    Сортировка слиянием & 0,004 & 0,007 & 0,014 & 0,022 & 0,029 & 0,048 \\ \hline
    Сортировка выбором & 1,064 & 2,986 & 11,876 & 26,528 & 48,269 & 107,033 \\ \hline
    \end{tabular}
\end{table}

\begin{table}[h]
\caption{Случайные данные}
\centering
  \begin{tabular}{ | l | l | l | l | l | l | l |}
  \hline
    Алгоритм & 30k & 50k & 100k & 150k & 200k & 300k \\ \hline
    Пузырьковая сортировка & 3,154 & 8,893 & 35,567 & 80,754 & 142,139 & 319,785 \\ \hline
    Шейкерная сортировка & 2,408 & 6,749 & 27,170 & 62,281 & 108,503 & 244,321 \\ \hline
    Сортировка Шелла & 2,132 & 5,085 & 23,144 & 48,592  & 94,249 & 209,012 \\ \hline
    Сортировка вставками & 0,632 & 1,740 & 6,981 & 15,498 & 27,328 & 62,970 \\ \hline
    Сортировка слиянием & 0,005 & 0,009 & 0,018 & 0,028 & 0,039 & 0,062 \\ \hline
    Сортировка выбором & 1,062 & 2,945 & 11,754 & 26,655 & 47,259 & 106,825 \\ \hline
    \end{tabular}
\end{table}

Как мы можем видеть из представленных данных, на частично упорядоченном массиве данных хорошо
показала себя сортировка вставками. Она не требует дополнительной памяти, отрабатывает за минимальное время.
Сортировка Шелла хоть и является её улучшением, показала более скромные результаты. Но этот алгоритм 
недостаточно изучен и допускает правку некоторых параметров, которые могут повлиять на его производительость.
Не самое плохое время работы продемонстрировали пузырьковые сортировки, несмотря на то, что их сложность
равна $O(N^2)$. А вот сортировка выбором - явный аутсайдер. Она требует больше всего времени.\par
Но все меняется при случайных данных. Сортировка слиянием показывает отличный результат. Сортировка выбором
так же продемонстрировала прежнее время работы. Отсюда можем сделать вывод, что структура данных мало влияет на
скорость работы этих двух сортировок. А вот другие сортировки сильно деградировали на неупорядоченном
массиве данных. В особенности пузырьковые сортировки.\par
Таким образом, в большинстве случаев стоит применять сортировку слиянием. Многие языки высокого уровня
используют её модификации в качестве алгоритмов сортировки по умолчанию. Но у скорости есть своя цена.
Сортировка слиянием требует дополнительной памяти. Если этот ресурс ограничен, стоит рассмотреть сортировку
вставками. На случайных данных она сильно уступает сортировке слиянием по времени, но гораздо более экономично
расходует память.