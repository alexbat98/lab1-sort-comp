\appendix 
\section*{Приложение} 
\phantomsection
\addcontentsline{toc}{section}{{\bf Приложение}} 
\markboth{{\bf Приложение}}{{\bf Приложение}}

\begin{lstlisting}
#include <stdlib.h>
#include <time.h>
#include <stdio.h>
#include <string.h>
#include "algorithms.h"
/**
 * Классическая пузырьковая сортировка
 * @param arr Массив значений для сортировки
 * @param n Длина массива
 */
void bubble_sort(int *arr, int n)
{
    int i, l, hasChanged;
    l = n - 1;
    do
    {
        hasChanged = 0;
        for (i = 0; i < l; ++i)
        {
            if (arr[i] > arr[i + 1])
            {
                arr[i] ^= arr[i + 1];
                arr[i + 1] ^= arr[i];
                arr[i] ^= arr[i + 1];
                hasChanged = 1;
            }
        }
        l--;
    } while (hasChanged);
}
/**
 * Улучшенная сортировка пузырьком с проходом в обе стороны
 * @param arr Массив данных
 * @param n Количество элементов
 */
void better_bubble_sort(int *arr, int n)
{
    int i, start, finish, hasChanged;
    start = 0;
    finish = n - 1;
    do
    {
        hasChanged = 0;
        for (i = start; i < finish; ++i)
        {
            if (arr[i] > arr[i + 1])
            {
                arr[i] ^= arr[i + 1];
                arr[i + 1] ^= arr[i];
                arr[i] ^= arr[i + 1];
                hasChanged = 1;
            }
        }
        --finish;
        if (hasChanged)
        {
            for (i = finish - 1; i >= start; --i)
            {
                if (arr[i] > arr[i + 1])
                {
                    arr[i] ^= arr[i + 1];
                    arr[i + 1] ^= arr[i];
                    arr[i] ^= arr[i + 1];
                    hasChanged = 1;
                }
            }
            start++;
        }
    }
    while (hasChanged);
}
/**
 * Сортировка Шелла
 * @param arr Массив данных
 * @param n Количество элементов
 */
void shell_sort(int *arr, int n)
{
    int i, hasChanged;
    int d = n;
    do
    {
        d = (d + 1) / 2;
        hasChanged = 0;
        for (i = 0; i < n - d; ++i)
        {
            if (arr[i] > arr[i + d])
            {
                arr[i] ^= arr[i+d];
                arr[i + d] ^= arr[i];
                arr[i] ^= arr[i + d];
                hasChanged = 1;
            }
        }
    }
    while (d != 1 || hasChanged);
}
/**
 * Поиск минимального элемента в массиве
 * @param arr Массив
 * @param n Количество элементов
 * @return Индекс минимального
 */
int min(int *arr, int n)
{
    int i, min_idx = 0;

    for (i = 0; i < n; ++i)
    {
        if (arr[i] < arr[min_idx])
        {
            min_idx = i;
        }
    }
    return min_idx;
}
/**
 * Сортировка выбором
 * @param arr Массив данных
 * @param n Количество элементов
 */
void selection_sort(int *arr, int n)
{
    int i, j, pos;
    for (i = 0; i < n - 1; ++i)
    {
        pos = i;
        for (j = i + 1; j < n; ++j)
        {
            if (arr[pos] > arr[j])
            {
                pos = j;
            }
        }
        if (pos != i)
        {
            arr[i] ^= arr[pos];
            arr[pos] ^= arr[i];
            arr[i] ^= arr[pos];
        }
    }
}
/**
 * Сортировка вставками
 * @param arr Массив
 * @param n Количество элементов
 */
void insertion_sort(int *arr, int n)
{
    int i, j, b;
    for (i = 0; i < n - 1; ++i)
    {
        b = arr[i + 1];
        j = i;
        while ((j >= 0) && (b < arr[j]))
        {
            arr[j + 1] = arr[j];
            j--;
        }
        arr[j + 1] = b;
    }
}
/**
 * Алгоритм слияния двух упорядоченных массивов
 * @param first Первая часть
 * @param nf Размер
 * @param second Вторая часть
 * @param ns Размер
 * @param result Результат
 * @param k Количество
 */
void merge ( int * first, int nf, int * second, int ns, int * result, int k )
{
    int count = 0, i = 0, j = 0;
    first[nf] = INT_MAX;
    second[ns] = INT_MAX;
    while (count < nf + ns)
    {
        if (first[i] < second[j])
        {
            result[k + count] = first[i++];
            count++;
        } else
        {
            result[k + count] = second[j++];
            count++;
        }
    }
}
/**
 * Сортировка слиянием
 * @param arr Исходный массив
 * @param n Количество элементов
 */
void merge_sort ( int * arr, int n )
{
    int i;
    int h = 1;
    int begin;
    int nf, ns;
    int *first, *second;
    first = (int *) calloc (n, sizeof(int));
    second = (int *) calloc (n / 2 + 1, sizeof(int));
    while (h < n)
    {
        begin = 0;
        while (begin < n - 1)
        {
            nf = 0;
            for (i = 0; (i < h) && (begin + i < n); i++)
            {
                first[i] = arr[begin + i];
                nf++;
            }
            ns = 0;
            for (i = 0; (i < h) && (begin + h + i < n); i++)
            {
                second[i] = arr[begin + h + i];
                ns++;
            }
            merge(first, nf, second, ns, arr, begin);
            begin += 2 * h;
        }
        h *= 2;
    }
}
\end{lstlisting}